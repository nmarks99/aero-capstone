\chapter{Future Work}
Although significant progress has been made on the controls side of the project, the ultimate goal of landing the rocket at a movable was not achieved. The three leading causes of this failure include the lack of infrastructure to safely test a landing system, insufficient thrust to weight ratio, and the time constraint for developing a GPS/vision based landing system.

As previously stated, given time constraints, the motors were not able to function. An obvious next step for the improvement of the current design is to get the motors with the proper amount of thrust to function in the rocket setup.

Moreover, another concern that the team had, with the current motors, was the low thrust to weight ratio. Currently, the thrust to weight ratio is 1.25. This ratio is slightly lower than the team would like for the rocket to quickly catch itself from free fall. Currently, the rocket will need to be dropped from a height of 15.68 meters (assuming a deployment time of 0.8 seconds). This is also assuming that the rotors would be constantly functioning at full thrust during the drop, and to improve stability, the team would like the rocket to be able to land without constantly functioning in a state of maximum thrust. Regardless, for performance optimization, two obvious steps would be to either increase the thrust from the propellers or decrease the weight, or both. 

In order to increase the thrust, the team recommends incorporating better motors and possibly investigating further into other propeller shapes.

Similarly, the weight of the rocket can be reduced further to increase the thrust-to-weight ratio. One possible way to do this is by using different materials for the 3-D printed components.  There are some structural components that do not take load; for these, lighter 3-D printed materials can be used. Also, material from structural parts that do not take much load can possibly be removed to reduce mass. However, the final design is pretty bare bones, so there is not much room for improvement in this regard. Moreover,  if the size of the legs is decreased, the mass of the legs will decrease by consequence, and as a result of the legs decreasing, the size of the rocket will be able to decrease in height. This will require less body tube, which is a significant source of mass.

A significant improvement to the current design that can be made is using lighter, customized springs. This will allow both the torsional springs (for arm deployment) and the linear springs (for leg deployment) to be optimized for performance. Ideally, the spring force is big enough to reliably deploy the limbs but not too big as to cause structural damage to components.

After the steps recommended above and further flight testing on the updated design have been completed, the final step would be to assemble the landing mechanism onto an actual amateur rocket. This will come with complications of its own. The changes in density of the atmosphere will cause problems that were irrelevant for this particular project but should be considered for a fully integrated rocket lander. Nonetheless, when the attachments to the rocket are able to reliably land the rocket at a desired location, the ultimate goal of this project would be achieved.