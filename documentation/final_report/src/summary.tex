\chapter{Executive Summary}
On December 21, 2015, SpaceX accomplished one of the greatest feats in modern engineering history -- they successfully landed a first stage Falcon 9 booster after an orbital flight. Vertical takeoff, vertical landing (VTVL) technology has enabled humans to access space more rapidly, reliably, and affordably than ever before. Fittingly, the interest of VTVL boosters has also trickled down into the hobbyist rocketry community. Currently, there are hobbyist rockets that are able to land using a parachute; however, there is no design that takes advantage of rotors. Our client, Professor Sridhar Krishnaswamy, saw potential for innovation in this space and tasked us with building a VTVL rotor-powered rocket lander to broaden the capabilities of amateur rocketry.

The team identified two primary subsystems that were developed in parallel and merged into a final alpha prototype. The controls subteam was responsible for detecting drop, generating thrust, flight controls, deployment controls, and stabilization. The structures subteam designed the accompanying hardware to support mounting, actuation, deployment, and landing of the system. 

The controls subteam built a drone from scratch using a Raspberry Pi as the flight computer and a Pixhawk microcontroller as the flight controller. In conjunction, they are able to delegate commands and stabilize the drone. The team heavily relied on ArduPilot, an open-source drone control and stabilization library which was communicated with through an open-source python library called DroneKit. The robustness, customization, and plug-and-play nature of these systems lent themselves well to integration with structural components down the line.

The structures subteam was responsible for selecting a rocket body, developing a reusable and maintainable core assembly, and designing stowed and deployed locking mechanism for the in-house developed motor arms and landing legs. Due to the large number of components and systems being developed concurrently throughout the project, the design process was highly iterative and testing oriented. 

The alpha prototype is a 41 inch tall and 6 inch in diameter phenolic cardboard rocket tube with an internal core structure that securely houses and facilitates actuation of all deployable components. With a 1.25 thrust-to-weight ratio and a deployment time of 0.8 seconds, the lander would require 15.68m of drop height to achieve hover and it was deemed unsafe to test with current facility limitations. The final prototype is capable of detecting drop and deploying arms and legs. All electronic hardware and control software exist to achieve stable landing.

This report is a technical summary of the design decisions, analysis, and testing performed to achieve the alpha prototype. After reading this report, one should feel comfortable replicating the work that has been done in a more efficient manner, and making necessary improvements to achieve a successful landing. 