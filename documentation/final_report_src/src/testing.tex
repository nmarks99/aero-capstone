\chapter{Testing \& Performance Evaluations}

Although the final design was not able to be thoroughly tested for flight, different aspects of the design were tested. Some of these tests include:
\begin{itemize} \item Deployment Time for the Arms \item Thrust To Weight Ratio \item Reliability to trigger deployment of arms and legs \item Reliability to lock arms in deployed positions
\item Reliability to lock legs in deployed positions \end{itemize}

This section of the report lays out the testing procedures done, the results, and discussion about relevant metrics.

\section{Deployment Time for the Arms}
As previously stated, the deployment time for the arms is a very critical part of the design. In order to reduce the amount of distance for the rocket to begin hovering, decreasing the deployment time as close to 0 seconds as possible is ideal. To test this, the deployment process for the final design was recorded 3 separate times with an iPhone. Note that the final design deployment releases both the arms and the legs simultaneously. 

The iPhone video was exported to a computer where the analysis was done. To find the deployment time, the video's FPS was determined (30 FPS). Then, for each video, the frames were counted from the time that the signal was sent to the servo to spin until the arms were in their locked position. Table \ref{table:deploymenttime} shows the trial number along with the number of frames and deployment time.

\begin{table}[H]
\centering
% \footnotesize
\caption{Deployment Time}
\label{table:deploymenttime}
\begin{tabular}{|
>{\raggedright\arraybackslash}p{.3\textwidth}|
>{\raggedright\arraybackslash}p{.3\textwidth}|
>{\raggedright\arraybackslash}p{.3\textwidth}|
}
    \hline
     \textbf{Trial} & \textbf{Number of Frames} & \textbf{Deployment Time in seconds}
    \\\hline 
     Trial 1 & 20 & 0.65
     \\\hline 
     Trial 2 & 21 & 0.70
     \\\hline
     Trial 3 & 23 & 0.76
     \\\hline
     Average & 21.33 & 0.711
    \\\hline
\end{tabular}
\end{table}

From this simple experiment, the deployment time was estimated to be about 0.7 seconds. To provide a safety factor for the deployment time, the team chose to look at the deployment time as a range; the lowest value was 0.65 second and the largest value was 0.80 seconds. This particular testing was done to inspect if metric \#5, deployment time, was met. The ideal value was 0 seconds, and the marginal value was 1 second. The deployment time, therefore, does meet the metric. 

\section{Thrust To Weight Ratio}
Another parameter that is extremely important to reaching the metrics is the thrust to weight ratio. This ratio will influence whether or not the rocket will be able to catch itself as explained in section \ref{thrust_analysis}. 
To test/measure this, the mass of the rocket needed to be determined. To do this, the team used a simple scale and the full assembly. The mass of the entire assembly was determined to be 3.2 kg. 

Moreover, the thrust generated by one propeller and motor combination was found to be about 10N as explained earlier in section \ref{design:motors-selection}. Assuming that the total thrust from the 4 propeller is the sum of the individual propellers, the total thrust generated is approximately 40N. This results in a thrust to weight ratio of approximately 1.25. 
In order to further improve the accuracy of the thrust to weight ratio, further testing would need to be done with all 4 propellers running simultaneously.

This particular test was done to figure out whether metrics \#1, metric \#4, and metric \#9 were met.

Metric \#1 is the distance it takes the design to decelerate to a hover. The ideal value is 0 meters while the marginal value is the 20 meters. Given the current thrust to weight ratio, this metric is satisfied because, given the deployment time, the current design only requires 15.68 meters to decelerate to a hover.

Metric \#4 is the landing velocity. The ideal value was 0.1 meters per second while the marginal value was 0.5 meters per second. Using the thrust to weight ratio, the landing velocity will be close to the ideal value given that the rocket is dropped from above 15.68 meters. The ability to land as close to the ideal value as possible cannot be determined because it depends on the ability to control the rocket, which, due to time constraints, was not tested.

Metric \#9 is the thrust to weight ratio. The ideal value was 4 while the marginal value was 1.01. Evidently, the thrust to weight ratio achieves this metric. However, increasing the thrust-to-weight ratio will help optimize performance.

\section{Deployment Reliability}
In addition to deployment time, it is critical that the arms and legs are able to reliably deploy. For a reminder of how the arms deploy, please refer to section \ref{pre-deployment_arm}; for a reminder of how the legs deploy, please refer to section \ref{legpredeploymentlock}.

To test the reliability of the deployment process, the team took a straight forward approach. Simply put, the deployment process was repeated multiple times. If the deployment process occurred, the trial was marked as successful. In addition to this, other observations were made. Table \ref{table:armdeploymentreliability} gives the results. Note that this test was done to see if the pre-deployment locking mechanisms worked well as well as the linear spring and torsion spring; testing to determine if the legs and arms lock in their deployed position is done in the next section.

\begin{table}[H]
\centering
% \footnotesize
\caption{Arm and Leg Deployment Reliability}
\label{table:armdeploymentreliability}
\begin{tabular}{|
>{\raggedright\arraybackslash}p{.3\textwidth}|
>{\raggedright\arraybackslash}p{.3\textwidth}|
>{\raggedright\arraybackslash}p{.3\textwidth}|
}
    \hline
     \textbf{Trial} & \textbf{Successful or Unsuccessful } & \textbf{Observations}
    \\\hline 
     Trial 1 & Successful & {None}
     \\\hline 
     Trial 2 & Successful & {None}
     \\\hline
     Trial 3 & Successful & {Placement of the latch was too close to the end; one arm deployed earlier}
     \\\hline
    Trial 4 & Successful & {None}
     \\\hline
    Trial 5 & Successful & {None}
    \\\hline
    Trial 6 & Successful & {None}
    \\\hline
\end{tabular}
\end{table}

The table shows that the team was successful in creating a design that reliably deploys the arm. From the 6 trials done, the arms deployed every single time. Unfortunately, one trial demonstrated that one arm deployed earlier than the others. This can cause flight concerns; in consequence, a future step would be to increase the reliability to deploy all 4 arms simultaneously.

This testing was done to test whether metric \#11 is met. The metric is the percentage of success to successfully deploy arms and legs; with the ideal value being 100\% deployment success while the marginal value being 90\% deployment success. From the acquired data, the team determined that this metric is met.

However, the other part of the testing was done to test whether metric \#11 was met. The ideal value was 100\% arm simultaneous deployment while the marginal value was 95\% arm simultaneous deployment. From the testing, the results showed that the arms simultaneously deployed approximately 83\% of the time. The team believes that this was due to tolerances on the arm design. As a future step, the team encourages to work on increasing the reliability of simultaneous arm deployment.

\section{Arm Locking Reliability}
In order to ensure that the arms were able to lock in place when deployed, the team initially considered using magnets. To test the reliability of this locking mechanism, the design was adjusted to incorporate the magnets. Then, the arms were deployed and observations were made. After 3 trials of the deployment process, the team quickly realized that the magnets were not efficiently locking the arms in their deployed position. Instead, the arms would oscillate when deployed and finally lock in position. These major oscillations are undesirable, so the team investigated other options.

The final design uses the locking mechanism explained in section \ref{armdeployedlockignmechanism}. To test this, a similar process was considered. The full assembly was created, and the deployment process was initiated by a user signal. Then, the team observed to see if the arms locked in their deployed position. Table \ref{table:armlockingreliability} shows the results.

\begin{table}[H]
\centering
% \footnotesize
\caption{Arm Locking Mechanism Reliability}
\label{table:armlockingreliability}
\begin{tabular}{|
>{\raggedright\arraybackslash}p{.3\textwidth}|
>{\raggedright\arraybackslash}p{.3\textwidth}|
>{\raggedright\arraybackslash}p{.3\textwidth}|
}
    \hline
     \textbf{Trial} & \textbf{Successful or Unsuccessful } & \textbf{Observations}
    \\\hline 
     Trial 1 & Successful & {None}
     \\\hline 
     Trial 2 & Successful & {None}
     \\\hline
     Trial 3 & Successful & {None}
     \\\hline
    Trial 4 & Successful & {None}
     \\\hline
    Trial 5 & Successful & {None}
    \\\hline
    Trial 6 & Successful & {None}
    \\\hline
    Trial 7 & Unsuccessful & {The torsional spring constant should be bigger }
    \\\hline
    Trial 8 & Successful & {None}
    \\\hline
    Trial 9 & Successful & {None}
    \\\hline
    Trial 10 & Unsuccessful & {The locking seems to work less with increasing attempts}
    \\\hline
\end{tabular}
\end{table}

This test was done to determine if metric \#12 was met. Metric \#12 is the percentage of success to lock arms in deployed position. The ideal value was 100\% while the marginal value was 95\%. Given our test results, the reliability is not very high. The arms locked 80\% of the time. 

The two trials which yielded unsuccessful results alerted the team of a design flaw. This design flaw was particular due to the torsional spring used in the design. A future step would be to optimize the torsional spring constant in order to ensure more reliability in the arm locking mechanism.

\section{Leg Locking Reliability}
In order to ensure that the legs properly work, the legs need to effectively lock in their position as elaborated in section \ref{legdeployedlockingmechanism}. The team used the full assembly and deployed the legs 10 times and recorded the information similar to the arm locking mechanism section. Table \ref{table:leglockingreliability} conveys the results.

\begin{table}[H]
\centering
% \footnotesize
\caption{Leg Locking Mechanism Reliability}
\label{table:leglockingreliability}
\begin{tabular}{|
>{\raggedright\arraybackslash}p{.3\textwidth}|
>{\raggedright\arraybackslash}p{.3\textwidth}|
>{\raggedright\arraybackslash}p{.3\textwidth}|
}
    \hline
     \textbf{Trial} & \textbf{Successful or Unsuccessful } & \textbf{Observations}
    \\\hline 
     Trial 1 & Successful & {None}
     \\\hline 
     Trial 2 & Successful & {None}
     \\\hline
     Trial 3 & Unsuccessful & {Spring needs to be fully extended}
     \\\hline
    Trial 4 & Successful & {None}
     \\\hline
    Trial 5 & Unsuccessful & {Not enough velocity to go past pins}
    \\\hline
    Trial 6 & Successful & {None}
    \\\hline
    Trial 7 & Successful & {None}
    \\\hline
    Trial 8 & Successful & {None}
    \\\hline
    Trial 9 & Unsuccessful & {None}
    \\\hline
    Trial 10 & Successful & {None}
    \\\hline
\end{tabular}
\end{table}

The three failed trials demonstrated lack of reliability in locking the legs in the deployed position. This was primarily due to design flaws with regards to the linear spring used as well as tolerance issues with the carriage and the spring pins. 

Metric \#13 is Percentage of success to lock legs in deployed position.The ideal value is 100\% while the marginal value is 95\%. Given the testing results, this metric was not met.

The three failed trials demonstrated lack of reliability in locking the legs in the deployed position. This was primarily due to design flaws with regards to the linear spring used as well as tolerance issues with the carriage and the spring pins. Possible improvements should come through adjustments to the carriage/ spring pins interference as well as the linear spring.