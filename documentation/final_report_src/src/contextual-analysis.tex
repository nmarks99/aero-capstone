\chapter{Contextual Analysis}
\section{Economic Impact}
Minimizing the cost to get to orbit is a major barrier to humans being able to do useful and interesting things in space. Coming in at a whopping \$16.87 per gallon, RP-1 (Rocket Propellant 1) is much more expensive than other types of fuel. At scale, a rotor based landing solution offers large cost savings over the lifetime of a reusable rocket. These cost savings are amplified by the ability to land at a spaceport within close proximity to a city as the cost of ‘last mile’ transport of the people and goods being carried by rockets will be less. Additionally, the introduction of a rotor based landing solution can create jobs in many places around the world as well as other economic opportunities. 

Overall, through a simplification of the landing process, the rocket company may be able to save a significant amount of money on a few parts. The first being less fuel consumption since the rotor lift is produced via force on the blades rather than a combustion set-up which involves high amounts of propellant. Along with that, assuming this process of landing proves effective, any percentage improvements of success land rates could improve the risk that a company takes on every time it attempts to land a multi-million dollar device such as the rocket. Finally, with this being a possible alternative to land within a city, the company will not have to purchase necessary permits and create the required vehicles and machinery to set-up a barge landing in the ocean or at a land landing area.

\section{Global Impact}
There are numerous geographical considerations to ponder when it comes to launching rockets. It is preferable to launch a rocket from the equator so they can take advantage of the increase in Earth’s rotational speed. This gives countries on or near the equator a significant advantage when it comes to putting payloads in space. Additionally, it is preferable to launch in areas near the ocean or in a vast open area that can be cleared within a few mile radius of the launch pad as a safety precaution. While launch locations will likely always need to be cleared due to the volatility of rocket propellant, a rotor based landing solution opens the possibility of landing at a spaceport with better proximity to an urban environment. This capability will be important if spaceflight will one day be used to transport cargo and people around the world. 
Globally, the ability to land rockets with rotors expands the opportunity for space travel; this can, in the long run, help humans in its mission to become a multi-planetary species. This is less of an issue when it comes to the landing process as the rocket landing system should be able to handle most environments and is not as restricted by location.

\section{Cultural Impact}
On the cultural side, space is only within reach for people and governments with considerable financial resources and significant spending capabilities. We saw this in the Apollo era, and we are seeing it again today with the interest of billionaires bolstering the private space sector. Unfortunately, this will likely create more economic disparity between the upper class and lower class. On the other hand, the continuous interest in space travel leads to international collaborations, paving the way for an multiplanetary future through a common goal. However, this can also create aggressive competition between different nations and private space industries like we saw during the Space Race (and are seeing today between the US, China, and Russia). In addition to that, this field is only capable of having the participants be companies / governments that are capable of expending high amounts of capital to stay competitive. 

If  rockets are being used to move humans (or flying in a region which poses danger to humans in any way), safety must be paramount. Luckily, safety is also in the mind of the rocket manufacturer at they likely want to protect their many million dollar investment. 

\section{Environmental Impact}
The list of environmental concerns directly associated with our project have to do with the materials used by our rockets. For example, lithium polymer batteries can cause surface water contamination if not adequately disposed of. Other materials such as carbon fiber, silicon chips, and ABS can be toxic if not disposed of correctly. This ultimately leads to contamination of the local environment. For this reason, the final product should be disposed of correctly to avoid any contamination after its life cycle. Moreover, during the class, we have acquired many items from all over the world, leaving behind CO2 emissions and extra, unnecessary packaging. Another important environmental consideration is what living things are at risk when testing is done. We do not want to be responsible for humans being injured or other living things such as birds. Because our project is intended to be a proof of concept, an analysis of environmental impact of the scaled-up version is applicable. The rocket lander will essentially create less waste because more parts of the rocket will be reusable; additionally, a rocket able to land with rotors will decrease the use of fuel, associated with launch and re-entry, or the incineration of payloads re-entering the earth’s atmosphere from LEO – it’s bad news all around.

Since our project is a proof of concept for a possible landing system that is more efficient than the typical combustion landing propellant method along with being possibly able to land in a more urban setting, perhaps in the future this will be more effective and minimize some of the current massive detrimental effects to the environment.

\section{Societal Impact}
On a more positive note, the reason we put up with the negative implications of spaceflight is the promise of what it offers us. Most significantly, spaceflight is the only path to making humans multiplanetary. Being a multiplanetary species is important for two reasons:

\begin{enumerate}
    \item Because expanding our scientific knowledge and capabilities is important to continually adapting and growing as a species.
    \item To protect us in the case of an environmental anomaly (Gamma ray burst, asteroid, global warming, this gets depressing…not a matter of if but when.) 
\end{enumerate}

The private space industry is in the process of exciting an entire generation of engineers, scientists, mathematicians, astronomers, and any other STEM discipline you can think of. It offers a vast number of high-paying and highly technical jobs and the potential for interplanetary movement of people and goods in earth shattering times. Within the past 100 years we have already seen the profound impact of space technologies and their impact on our world. GPS, credit card transactions, and integrated circuits would not exist without aerospace. The next 100 are going to be a treat. 

On the smaller scale of our group, this project has given us perspective on the aerospace industry and experience building an aerospace vehicle from scratch. We will all be going into the aerospace industry and having this knowledge about technology, companies, people, and ethics can provide us with the tools necessary to positively build upon the field. 
